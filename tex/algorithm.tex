\documentclass[UTF8]{ctexart}
    \usepackage{ctex}
\begin{document}

\section{stack}
\subsection{函数栈}
函数调用的局部状态之所以用栈来记录是因为这些数据的存活时间满足“后入先出”(LIFO)顺序,
而栈的基本操作正好就是支持这种顺序的访问。

\section{图灵机}
\subsection{图灵机是为了解决希尔伯特问题}

\section{greatest common divisor}

\section{数据结构}
算法需要组织数据,组织数据就产生了数据结构

\section{解决典型问题的各种有效算法}

\section{排序}
\subsection{插入排序}
\subsection{选择排序}
\subsection{希尔排序}
\subsection{快速排序}
\subsection{归并排序}
\subsection{堆排序}

\section{查找}
\subsection{二叉查找树}
\subsection{平衡查找树}
\subsection{散列表}

\section{map}

\section{string}

\section{编程用到的7种语法}
\subsection{原始数据类型}
int,float,bool
\subsection{语句}
声明、赋值、条件、循环、调用、返回
\subsection{数组}
array is 多个同种数据类型的值的集合
\subsection{function method}

\end{document}