\documentclass[oneside,12pt,twiside,a4paper]{ctexbook}

\usepackage{ctex}
\usepackage{listings} % 插入代码

%
% 插入代码
%
\lstset{language=C++}%这条命令可以让LaTeX排版时将C++ 键字突出显示
\lstset{breaklines}%这条命令可以让LaTeX自动将长的代码行换行排版
\lstset{extendedchars=false}%这一条命令可以解决代码跨页时,章节标题,页眉等汉字不显示的问题
\lstset{
    numbers=left,
    numberstyle= \tiny,
    keywordstyle= \color{ blue!70},
    commentstyle= \color{red!50!green!50!blue!50},
    % frame=shadowbox, % 阴影效果
    rulesepcolor= \color{ red!20!green!20!blue!20} ,
    escapeinside=``, % 英文分号中可写入中文
    xleftmargin=2em,xrightmargin=2em, aboveskip=1em,
    framexleftmargin=2em,
}

\usepackage{geometry} %页边距和页眉页脚
\usepackage{titlesec}%TEX标题与正文间距,标题与上下文距离调整
\titleformat{\chapter}[display]{\normalfont\huge\bfseries\center}{\chaptertitlename}{1pt}{\Huge}
\titleformat{\section}{\normalfont\Large\bfseries}{\thesection}{1em}{}
\titleformat{\subsection}{\normalfont\large\bfseries}{\thesubsection}{1em}{}
\titleformat{\subsubsection}{\normalfont\normalsize\bfseries}{\thesubsubsection}{1em}{}
\titleformat{\paragraph}[runin]{\normalfont\normalsize\bfseries}{\theparagraph}{1em}{}
\titleformat{\subparagraph}[runin]{\normalfont\normalsize\bfseries}{\thesubparagraph}{1em}{}
\titlespacing*{\chapter} {0pt}{10pt}{10pt}
\titlespacing*{\section} {0pt}{0.5ex plus 1ex minus .2ex}{0.3ex plus .2ex}
\titlespacing*{\subsection} {0pt}{0.25ex plus 1ex minus .1ex}{0.5ex plus .1ex}
\titlespacing*{\subsubsection}{0pt}{3.25ex plus 1ex minus .2ex}{1.5ex plus .2ex}
\titlespacing*{\paragraph} {0pt}{3.25ex plus 1ex minus .2ex}{1em}
\titlespacing*{\subparagraph} {\parindent}{3.25ex plus 1ex minus .2ex}{1em}
\geometry{left=2.0cm,right=2.1cm,top=2.0cm,bottom=2.5cm}% 页边距和页眉页脚


\begin{document}

\author
{
何书文\\
1201220707@pku.edu.cn\\
}

%%%%%%%%%%%%%%%%%%%%%%%%%%%%%%%%%%%%%%%%%%%%%%%%%%%%%%%%%%%%%%%%%%%%%%%%%%%%%%%%%%%%%%%%%%%%%%%%%%%%%%%%%%
\title{algorithm}
\maketitle
\tableofcontents
%页眉

%%%%%%%%%%%%%%%%%%%%%%%%%%%%%%%%%%%%%%%%%%%%%%%%%%%%%%%%%%%%%%%%%%%%%%%%%%%%%%%%%%%%%%%%%%%%%%%%%%%%%%%%%%
%\rhead{\includegraphics{   shuwen.png}}
%\rhead{}
%%%%%%%%%%%%%%%%%%%%%%%%%%%%%%%%%%%%%%%%%%%%%%%%%%%%%%%%%%%%%%%%%%%%%%%%%%%%%%%%%%%%%%%%%%%%%%%%%%%%%%%%%%

\chapter{list}
\section{list}
\subsection{addTwoNumbers}
给出两个非空的链表用来表示两个非负的整数。其中,它们各自的位数是按照逆序的方式存储的,并且它们的每个节点只能存储一位数字。
如果,我们将这两个数相加起来,则会返回一个新的链表来表示它们的和。您可以假设除了数字0之外,这两个数都不会以0开头。

示例:
输入:(2 -> 4 -> 3) + (5 -> 6 -> 4)

输出:7 -> 0 -> 8

原因:342 + 465 = 807

\begin{lstlisting}

    package main
    
    import (
        "fmt"
    )
    
    type ListNode struct {
        Val  int
        Next *ListNode
    }
    type List struct {
        headNode *ListNode // head node
    }
    
    // 1.Insert
    func Insert(value int, list *ListNode, position *ListNode) {
        tempCell := new(ListNode)
        if tempCell == nil {
            fmt.Println("out of space")
        }
        tempCell.Val = value
        tempCell.Next = position.Next
        position.Next = tempCell
    }
    
    // 2.Print
    func PrintList(list *ListNode) {
        if list.Next != nil {
            fmt.Println(list.Val)
            PrintList(list.Next)
        } else {
            fmt.Println(list.Val)
        }
    }
    
    func main() {
        l1 := new(ListNode)
        listDate := l1
        // insert data to l1
        Insert(2, listDate, l1)
        Insert(4, listDate, l1)
        Insert(3, listDate, l1)
        l2 := new(ListNode)
        //
        listDate2 := l2
        // insert data to l1
        Insert(5, listDate2, l2)
        Insert(6, listDate2, l2)
        Insert(4, listDate2, l2)
        l3 := addTwoNumbers(l1, l2)
        PrintList(l3)
    }
    
    func addTwoNumbers(l1 *ListNode, l2 *ListNode) *ListNode {
        promotion := 0     // 进位值, 只可能为0或1
        var head *ListNode // 结果表的头结点
        var rear *ListNode // 保存结果表的尾结点
        for nil != l1 || nil != l2 {
            sum := 0
            if nil != l1 {
                sum += l1.Val
                l1 = l1.Next
            }
            if nil != l2 {
                sum += l2.Val
                l2 = l2.Next
            }
    
            sum += promotion
            promotion = 0
    
            if sum >= 10 {
                promotion = 1
                sum = sum % 10
            }
    
            node := &ListNode{
                sum,
                nil,
            }
    
            if nil == head {
                head = node
                rear = node
            } else {
                rear.Next = node
                rear = node
            }
        }
    
        if promotion > 0 {
            rear.Next = &ListNode{
                promotion,
                nil,
            }
        }
        return head
    }
        
\end{lstlisting}

\end{document}