\documentclass[oneside,12pt,twiside,a4paper]{ctexbook}

\usepackage{ctex}
\usepackage{listings} % 插入代码

%
% 插入代码
%
\lstset{language=C++}%这条命令可以让LaTeX排版时将C++ 键字突出显示
\lstset{breaklines}%这条命令可以让LaTeX自动将长的代码行换行排版
\lstset{extendedchars=false}%这一条命令可以解决代码跨页时,章节标题,页眉等汉字不显示的问题
\lstset{
    numbers=left,
    numberstyle= \tiny,
    keywordstyle= \color{ blue!70},
    commentstyle= \color{red!50!green!50!blue!50},
    % frame=shadowbox, % 阴影效果
    rulesepcolor= \color{ red!20!green!20!blue!20} ,
    escapeinside=``, % 英文分号中可写入中文
    xleftmargin=2em,xrightmargin=2em, aboveskip=1em,
    framexleftmargin=2em,
}

\usepackage{geometry} %页边距和页眉页脚
\usepackage{titlesec}%TEX标题与正文间距,标题与上下文距离调整
\titleformat{\chapter}[display]{\normalfont\huge\bfseries\center}{\chaptertitlename}{1pt}{\Huge}
\titleformat{\section}{\normalfont\Large\bfseries}{\thesection}{1em}{}
\titleformat{\subsection}{\normalfont\large\bfseries}{\thesubsection}{1em}{}
\titleformat{\subsubsection}{\normalfont\normalsize\bfseries}{\thesubsubsection}{1em}{}
\titleformat{\paragraph}[runin]{\normalfont\normalsize\bfseries}{\theparagraph}{1em}{}
\titleformat{\subparagraph}[runin]{\normalfont\normalsize\bfseries}{\thesubparagraph}{1em}{}
\titlespacing*{\chapter} {0pt}{10pt}{10pt}
\titlespacing*{\section} {0pt}{0.5ex plus 1ex minus .2ex}{0.3ex plus .2ex}
\titlespacing*{\subsection} {0pt}{0.25ex plus 1ex minus .1ex}{0.5ex plus .1ex}
\titlespacing*{\subsubsection}{0pt}{3.25ex plus 1ex minus .2ex}{1.5ex plus .2ex}
\titlespacing*{\paragraph} {0pt}{3.25ex plus 1ex minus .2ex}{1em}
\titlespacing*{\subparagraph} {\parindent}{3.25ex plus 1ex minus .2ex}{1em}
\geometry{left=2.0cm,right=2.1cm,top=2.0cm,bottom=2.5cm}% 页边距和页眉页脚


\begin{document}

\author
{
Shuwen He\\
1201220707@pku.edu.cn\\
Written in Peking University Library
}

%%%%%%%%%%%%%%%%%%%%%%%%%%%%%%%%%%%%%%%%%%%%%%%%%%%%%%%%%%%%%%%%%%%%%%%%%%%%%%%%%%%%%%%%%%%%%%%%%%%%%%%%%%
\title{Algorithm}
\maketitle
\tableofcontents
%页眉

%%%%%%%%%%%%%%%%%%%%%%%%%%%%%%%%%%%%%%%%%%%%%%%%%%%%%%%%%%%%%%%%%%%%%%%%%%%%%%%%%%%%%%%%%%%%%%%%%%%%%%%%%%
%\rhead{\includegraphics{   shuwen.png}}
%\rhead{}
%%%%%%%%%%%%%%%%%%%%%%%%%%%%%%%%%%%%%%%%%%%%%%%%%%%%%%%%%%%%%%%%%%%%%%%%%%%%%%%%%%%%%%%%%%%%%%%%%%%%%%%%%%

%
\chapter{Algorithm}
\section{Book}
discrete mathematics

Abstract algebra

Ordinary differential equation

Mathematical analysis

Probability statistics

Computer software and theory

English

\section{Overview}
In this book I define algorithms as a branch of mathematics

Algorithm must be designed to solve practical problems. An algorithm that cannot solve practical problems is a vase.

Divide the algorithm in a way that solves the problem

Please believe that learning algorithms have methods

Fresh and refined, immediate, don't be sloppy

I proved the algorithmic mathematical kilometers system
%
\chapter{Logarithm}
\section{lgn}
$lgn = log_2n$ (Base 2 logarithm)

$lnn=log_en$ (Natural logarithm)

$lg^kn=(lgn)^k$ (Exponentiation)

$lglgn = lg(lgn)$ (complex)

%
\chapter{Stack}
stack : push + pop

%
\chapter{Queue}
queue : enqueue + dequeue

%
\chapter{list}
double linked list:value + next and prev
\section{list}
\begin{lstlisting}
    package main

    import(
        "fmt"
    )
    
    type ListNode struct{
        Val int 
        Next *ListNode
    }
    
    func main1()  {
        headNode := &ListNode{}
        listData := headNode
    
        InsertTail(1,listData,headNode)
        PrintList(listData)
    
        InsertTail(2,listData,headNode)
        PrintList(listData)
    
        InsertTail(3,listData,headNode)
        PrintList(listData)
    }
    
    func InsertTail(value int,list,position *ListNode)  {
        tempCell := new(ListNode)
    
        if tempCell == nil{
            fmt.Println("out of space")
        }
    
        tempCell.Val = value
        tempCell.Next = position.Next
        position.Next = tempCell
    }
    
    func PrintList(list *ListNode)  {
        if list.Next != nil{
            fmt.Print(list.Val,"->")
            PrintList(list.Next)
        }else{
            fmt.Println(list.Val)
        }
    }
    
    // 给出两个非空的链表用来表示两个非负的整数。其中,它们各自的位数是按照逆序的方式存储的,
    // 并且它们的每个节点只能存储一位数字。
    // 如果,我们将这两个数相加起来,则会返回一个新的链表来表示它们的和。
    // 您可以假设除了数字0之外,这两个数都不会以0开头。
    // 示例:
    // 输入:(2 -> 4 -> 3) + (5 -> 6 -> 4)
    // 输出:7 -> 0 -> 8
    // 原因:342 + 465 = 807
    
    // type ListNode2 struct {
    // 	Val  int
    // 	Next *ListNode2
    // }
    
    type List struct {
        headNode2 *ListNode // head node
    }
    
    func Insert2(value int, list *ListNode, position *ListNode) {
        tempCell := new(ListNode)
        if tempCell == nil {
            fmt.Println("out of space")
        }
        tempCell.Val = value
        tempCell.Next = position.Next
        position.Next = tempCell
    }
    
    func PrintList2(list *ListNode) {
        if list.Next != nil {
            fmt.Println(list.Val)
            PrintList2(list.Next)
        } else {
            fmt.Println(list.Val)
        }
    }
    
    func main() {
        l1 := new(ListNode)
        listDate := l1
        // insert data to l1
        Insert2(9, listDate, l1)
        Insert2(7, listDate, l1)
        Insert2(5, listDate, l1)
        l2 := new(ListNode)
        //
        listDate2 := l2
        // insert data to l1
        Insert2(4, listDate2, l2)
        Insert2(2, listDate2, l2)
        Insert2(8, listDate2, l2)
        l3 := addTwoNumbers(l1, l2)
        PrintList(l3)
    }
    
    func addTwoNumbers(l1 *ListNode, l2 *ListNode) *ListNode {
        promotion := 0     // 进位值, 只可能为0或1
        var head *ListNode // 结果表的头结点
        var rear *ListNode // 保存结果表的尾结点
        for nil != l1 || nil != l2 {
            sum := 0
            if nil != l1 {
                sum += l1.Val
                l1 = l1.Next
            }
            if nil != l2 {
                sum += l2.Val
                l2 = l2.Next
            }
    
            sum += promotion
            promotion = 0
    
            if sum >= 10 {
                promotion = 1
                sum = sum % 10
            }
    
            node := &ListNode{
                sum,
                nil,
            }
    
            if nil == head {
                head = node
                rear = node
            } else {
                rear.Next = node
                rear = node
            }
        }
    
        if promotion > 0 {
            rear.Next = &ListNode{
                promotion,
                nil,
            }
        }
        return head
    }    
\end{lstlisting}

%
\chapter{hash table}
\section{twoSum}
\begin{lstlisting}
// 给定一个整数数组nums和一个目标值target,
// 请你在该数组中找出和为目标值的那两个整数,并返回他们的数组下标。
// 你可以假设每种输入只会对应一个答案。但是,你不能重复利用这个数组中同样的元素。
// 示例:
// 给定 nums = [2, 7, 11, 15], target = 9
// 因为 nums[0] + nums[1] = 2 + 7 = 9
// 所以返回 [0, 1]

package main

import(
	"fmt"
)

func main()  {
	nums := []int{2,7,11,15}
	target := 9
	arr := twoSum(nums,target)
	fmt.Println("arr = ",arr)
}

func twoSum(nums []int, target int) []int {
    h := make(map[int]int)
    for i, value := range nums {
        if wanted, ok := h[value]; ok {
            return []int{wanted, i}
        } else {
            h[target-value] = i
        }
    }
    return nil
}
\end{lstlisting}

%
\chapter{divide and conquer}

%
\chapter{Search}
\section{Binary Search}
\section{Binary Search Tree}


%Sort
\chapter{Sort}
\section{comparison sort}
%
\subsection{quick sort}
quick sort (C. A. R. Hoare 1960) 

Divide and Conquer Algorithm for Quick Sort

Quick sort is better than heap sort
\begin{lstlisting}
    package main

    import (
        "math/rand"
    )
    
    // quick sort
    // 分治排序
    func main()  {
        var z []int
    
        for i := 0; i < 3; i++{
            z = append(z, rand.Intn(3))
        }
        
        quickSort(z)	
    }
    
    func quickSort(list []int)  {
        if len(list) <= 1{
            return
        }
    
        i, j := 0, len(list) - 1
        index := 1 // 第一次比较索引位置
        key := list[0] // 第一次比较参考值,选择第一个
    
        if list[index] > key{
            list[i], list[j] = list[j], list[i]
            j--
        }else{
            list[i], list[index] = list[index],list[i]
            i++
            index++
        }
    
        quickSort(list[:i]) // 处理参考值前面值
        quickSort(list[i+1:])
    }
\end{lstlisting}
%
\subsection{merge sort}
John von Neumann1945
Merge Sort:
The linear table to be sorted is continuously divided into several subtables until each 
subtable contains only one element. 
At this time, the subtable containing only one element can be considered as an ordered list.
Merge the subtables in pairs. Each time a new table is generated, 
a new and longer ordered list is generated. 
Repeat this step until there is only one subtable left. 
This subtable is an ordered linear table.
%
\subsection{heap sort}
$O(nlgn)$ (heap sort run time)

%
\section{bubble sort}
Repeated swapping of two adjacent elements in reverse order


%dynamic programming
\chapter{Dynamic programming}
\section{调度问题}
\section{矩阵链乘法}
\section{公共子序列}
\section{Binary search tree}

%
\chapter{greedy algorithm}
\section{Minimum spanning tree}
\section{Huffman coding}
Data compression technology (David A. Huffman1952MIT)

A Method for the Construction of Minimum-Redundancy Codes

%
\chapter{RBTree}
RBTree height is $O(lgn)$

color,key,left,right,p

\section{nature}

\section{rotate}

\end{document}